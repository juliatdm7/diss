% Options for packages loaded elsewhere
\PassOptionsToPackage{unicode}{hyperref}
\PassOptionsToPackage{hyphens}{url}
%
\documentclass[
]{article}
\usepackage{amsmath,amssymb}
\usepackage{iftex}
\ifPDFTeX
  \usepackage[T1]{fontenc}
  \usepackage[utf8]{inputenc}
  \usepackage{textcomp} % provide euro and other symbols
\else % if luatex or xetex
  \usepackage{unicode-math} % this also loads fontspec
  \defaultfontfeatures{Scale=MatchLowercase}
  \defaultfontfeatures[\rmfamily]{Ligatures=TeX,Scale=1}
\fi
\usepackage{lmodern}
\ifPDFTeX\else
  % xetex/luatex font selection
\fi
% Use upquote if available, for straight quotes in verbatim environments
\IfFileExists{upquote.sty}{\usepackage{upquote}}{}
\IfFileExists{microtype.sty}{% use microtype if available
  \usepackage[]{microtype}
  \UseMicrotypeSet[protrusion]{basicmath} % disable protrusion for tt fonts
}{}
\makeatletter
\@ifundefined{KOMAClassName}{% if non-KOMA class
  \IfFileExists{parskip.sty}{%
    \usepackage{parskip}
  }{% else
    \setlength{\parindent}{0pt}
    \setlength{\parskip}{6pt plus 2pt minus 1pt}}
}{% if KOMA class
  \KOMAoptions{parskip=half}}
\makeatother
\usepackage{xcolor}
\usepackage[margin=1in]{geometry}
\usepackage{color}
\usepackage{fancyvrb}
\newcommand{\VerbBar}{|}
\newcommand{\VERB}{\Verb[commandchars=\\\{\}]}
\DefineVerbatimEnvironment{Highlighting}{Verbatim}{commandchars=\\\{\}}
% Add ',fontsize=\small' for more characters per line
\usepackage{framed}
\definecolor{shadecolor}{RGB}{248,248,248}
\newenvironment{Shaded}{\begin{snugshade}}{\end{snugshade}}
\newcommand{\AlertTok}[1]{\textcolor[rgb]{0.94,0.16,0.16}{#1}}
\newcommand{\AnnotationTok}[1]{\textcolor[rgb]{0.56,0.35,0.01}{\textbf{\textit{#1}}}}
\newcommand{\AttributeTok}[1]{\textcolor[rgb]{0.13,0.29,0.53}{#1}}
\newcommand{\BaseNTok}[1]{\textcolor[rgb]{0.00,0.00,0.81}{#1}}
\newcommand{\BuiltInTok}[1]{#1}
\newcommand{\CharTok}[1]{\textcolor[rgb]{0.31,0.60,0.02}{#1}}
\newcommand{\CommentTok}[1]{\textcolor[rgb]{0.56,0.35,0.01}{\textit{#1}}}
\newcommand{\CommentVarTok}[1]{\textcolor[rgb]{0.56,0.35,0.01}{\textbf{\textit{#1}}}}
\newcommand{\ConstantTok}[1]{\textcolor[rgb]{0.56,0.35,0.01}{#1}}
\newcommand{\ControlFlowTok}[1]{\textcolor[rgb]{0.13,0.29,0.53}{\textbf{#1}}}
\newcommand{\DataTypeTok}[1]{\textcolor[rgb]{0.13,0.29,0.53}{#1}}
\newcommand{\DecValTok}[1]{\textcolor[rgb]{0.00,0.00,0.81}{#1}}
\newcommand{\DocumentationTok}[1]{\textcolor[rgb]{0.56,0.35,0.01}{\textbf{\textit{#1}}}}
\newcommand{\ErrorTok}[1]{\textcolor[rgb]{0.64,0.00,0.00}{\textbf{#1}}}
\newcommand{\ExtensionTok}[1]{#1}
\newcommand{\FloatTok}[1]{\textcolor[rgb]{0.00,0.00,0.81}{#1}}
\newcommand{\FunctionTok}[1]{\textcolor[rgb]{0.13,0.29,0.53}{\textbf{#1}}}
\newcommand{\ImportTok}[1]{#1}
\newcommand{\InformationTok}[1]{\textcolor[rgb]{0.56,0.35,0.01}{\textbf{\textit{#1}}}}
\newcommand{\KeywordTok}[1]{\textcolor[rgb]{0.13,0.29,0.53}{\textbf{#1}}}
\newcommand{\NormalTok}[1]{#1}
\newcommand{\OperatorTok}[1]{\textcolor[rgb]{0.81,0.36,0.00}{\textbf{#1}}}
\newcommand{\OtherTok}[1]{\textcolor[rgb]{0.56,0.35,0.01}{#1}}
\newcommand{\PreprocessorTok}[1]{\textcolor[rgb]{0.56,0.35,0.01}{\textit{#1}}}
\newcommand{\RegionMarkerTok}[1]{#1}
\newcommand{\SpecialCharTok}[1]{\textcolor[rgb]{0.81,0.36,0.00}{\textbf{#1}}}
\newcommand{\SpecialStringTok}[1]{\textcolor[rgb]{0.31,0.60,0.02}{#1}}
\newcommand{\StringTok}[1]{\textcolor[rgb]{0.31,0.60,0.02}{#1}}
\newcommand{\VariableTok}[1]{\textcolor[rgb]{0.00,0.00,0.00}{#1}}
\newcommand{\VerbatimStringTok}[1]{\textcolor[rgb]{0.31,0.60,0.02}{#1}}
\newcommand{\WarningTok}[1]{\textcolor[rgb]{0.56,0.35,0.01}{\textbf{\textit{#1}}}}
\usepackage{graphicx}
\makeatletter
\def\maxwidth{\ifdim\Gin@nat@width>\linewidth\linewidth\else\Gin@nat@width\fi}
\def\maxheight{\ifdim\Gin@nat@height>\textheight\textheight\else\Gin@nat@height\fi}
\makeatother
% Scale images if necessary, so that they will not overflow the page
% margins by default, and it is still possible to overwrite the defaults
% using explicit options in \includegraphics[width, height, ...]{}
\setkeys{Gin}{width=\maxwidth,height=\maxheight,keepaspectratio}
% Set default figure placement to htbp
\makeatletter
\def\fps@figure{htbp}
\makeatother
\setlength{\emergencystretch}{3em} % prevent overfull lines
\providecommand{\tightlist}{%
  \setlength{\itemsep}{0pt}\setlength{\parskip}{0pt}}
\setcounter{secnumdepth}{-\maxdimen} % remove section numbering
\ifLuaTeX
  \usepackage{selnolig}  % disable illegal ligatures
\fi
\usepackage{bookmark}
\IfFileExists{xurl.sty}{\usepackage{xurl}}{} % add URL line breaks if available
\urlstyle{same}
\hypersetup{
  pdftitle={Data preparation},
  pdfauthor={Julia},
  hidelinks,
  pdfcreator={LaTeX via pandoc}}

\title{Data preparation}
\author{Julia}
\date{2025-08-12}

\begin{document}
\maketitle

\subsection{Brief comment}\label{brief-comment}

This markdown document was created to briefly show the outputs of the
code included in the \texttt{dataset\_preparation.R} as the full data
will not be uploaded for confidentiality reasons. Therefore, this file
will have the exact same code with small modifications to show that each
step is doing.

\subsubsection{1. Loading packages and baseline data
sets}\label{loading-packages-and-baseline-data-sets}

\begin{Shaded}
\begin{Highlighting}[]
\CommentTok{\# Loading packages}

\FunctionTok{library}\NormalTok{(dplyr)  }
\end{Highlighting}
\end{Shaded}

\begin{verbatim}
## 
## Adjuntando el paquete: 'dplyr'
\end{verbatim}

\begin{verbatim}
## The following objects are masked from 'package:stats':
## 
##     filter, lag
\end{verbatim}

\begin{verbatim}
## The following objects are masked from 'package:base':
## 
##     intersect, setdiff, setequal, union
\end{verbatim}

\begin{Shaded}
\begin{Highlighting}[]
\FunctionTok{library}\NormalTok{(tidyverse)}
\end{Highlighting}
\end{Shaded}

\begin{verbatim}
## Warning: package 'ggplot2' was built under R version 4.4.3
\end{verbatim}

\begin{verbatim}
## Warning: package 'purrr' was built under R version 4.4.3
\end{verbatim}

\begin{verbatim}
## -- Attaching core tidyverse packages ------------------------ tidyverse 2.0.0 --
## v forcats   1.0.0     v readr     2.1.5
## v ggplot2   3.5.2     v stringr   1.5.1
## v lubridate 1.9.4     v tibble    3.2.1
## v purrr     1.0.4     v tidyr     1.3.1
\end{verbatim}

\begin{verbatim}
## -- Conflicts ------------------------------------------ tidyverse_conflicts() --
## x dplyr::filter() masks stats::filter()
## x dplyr::lag()    masks stats::lag()
## i Use the conflicted package (<http://conflicted.r-lib.org/>) to force all conflicts to become errors
\end{verbatim}

\begin{Shaded}
\begin{Highlighting}[]
\CommentTok{\# Loading baseline data sets}

\NormalTok{alladults }\OtherTok{\textless{}{-}} \FunctionTok{read.csv}\NormalTok{(}\StringTok{"C:/Users/julia/OneDrive {-} University of Edinburgh/EEB\_MSc\_UEd/Dissertation/diss/data/Adults\_2025.csv"}\NormalTok{)  }\CommentTok{\# loading all adult blue tits data}
\FunctionTok{as\_tibble}\NormalTok{(alladults)  }\CommentTok{\# visualising first rows of data}
\end{Highlighting}
\end{Shaded}

\begin{verbatim}
## # A tibble: 3,928 x 16
##    ring     year site    box   age sex    wing tarsus  mass  date time    ringer
##    <chr>   <int> <chr> <int> <int> <chr> <dbl>  <dbl> <dbl> <dbl> <chr>   <chr> 
##  1 Z632906  2015 ALN       3     6 F        59     NA  10.6   161 "13:45" IBC   
##  2 Z632905  2015 ALN       3     6 M        66     NA  10.6   161 "13:45" IBC   
##  3 Z632133  2015 ALN       4     6 F        62     NA  10.2   151 ""      JDS   
##  4 Z632134  2015 ALN       4     6 M        68     NA  11.9   151 ""      JDS   
##  5 Z632769  2015 ALN       5     5 F        63     NA  11.4   157 "16:00" JDS   
##  6 Z632770  2015 ALN       5     5 M        65     NA  10.5   157 "16:00" JDS   
##  7 Z632133  2016 ALN       2     6 F        63     NA  10.3   140 "15:30" JDS   
##  8 Z632838  2016 ALN       5     6 F        64     NA  10.7   148 "16:30" JDS   
##  9 Z632855  2016 ALN       6     6 F        60     NA  10.2   156 "14:10" IBC   
## 10 Z632856  2016 ALN       6     6 M        65     NA  10.5   156 "14:10" IBC   
## # i 3,918 more rows
## # i 4 more variables: season <chr>, history <int>, notes <chr>, catching <chr>
\end{verbatim}

\begin{Shaded}
\begin{Highlighting}[]
\NormalTok{allbirdphen }\OtherTok{\textless{}{-}} \FunctionTok{read.csv}\NormalTok{(}\StringTok{"C:/Users/julia/OneDrive {-} University of Edinburgh/EEB\_MSc\_UEd/Dissertation/diss/data/Bird\_Phenology\_2025.csv"}\NormalTok{)  }\CommentTok{\# loading all phenology data}
\FunctionTok{as\_tibble}\NormalTok{(allbirdphen)  }\CommentTok{\# visualising first rows of data}
\end{Highlighting}
\end{Shaded}

\begin{verbatim}
## # A tibble: 3,594 x 51
##     year site    box preadtor_proof species visit.frequency moss  n1    nl   
##    <int> <chr> <int> <chr>          <chr>   <chr>           <chr> <chr> <chr>
##  1  2014 FOF       1 ""             "bluti" ""              ""    "98"  "106"
##  2  2014 FOF       2 ""             "bluti" ""              ""    "118" "122"
##  3  2014 FOF       3 ""             "bluti" ""              ""    "82"  "98" 
##  4  2014 FOF       4 ""             "bluti" ""              ""    "90"  "106"
##  5  2014 FOF       5 ""             "bluti" ""              ""    "98"  "108"
##  6  2014 FOF       6 ""             "bluti" ""              ""    "98"  "102"
##  7  2014 BAD       1 ""             "bluti" ""              ""    "96"  "106"
##  8  2014 BAD       2 ""             ""      ""              ""    ""    ""   
##  9  2014 BAD       3 ""             "bluti" ""              ""    "100" "106"
## 10  2014 BAD       4 ""             ""      ""              ""    ""    ""   
## # i 3,584 more rows
## # i 42 more variables: latestfed <int>, latestcc <int>, fed <int>, cc <chr>,
## #   cs <int>, lastnotinc <chr>, fki <int>, clutch.swap.treatment <chr>,
## #   extra.eggs <int>, hd_1.45 <chr>, hatching_first_recorded <int>,
## #   hatching <chr>, number.hatched <chr>, weight <chr>, av.weight <chr>,
## #   time.of.day.for.hatch.weight <chr>, uhe <chr>, v1date <int>, v1alive <int>,
## #   v1poo <int>, v1time <chr>, v1duration <chr>, v2date <int>, ...
\end{verbatim}

\subsubsection{2. Selecting variables that will be included in the final
data
sets}\label{selecting-variables-that-will-be-included-in-the-final-data-sets}

Since we aim to study breeding success trends in blue tit populations
across age groups and across sites, we will extract those variables that
may be useful to study breeding success: first egg lay date
(\texttt{fed}), clutch size ('cs`) and number of fledgelings (suc).

\begin{Shaded}
\begin{Highlighting}[]
\CommentTok{\# Selecting variables that will be useful for the project from the phenology data}

\NormalTok{blutiphen }\OtherTok{\textless{}{-}}\NormalTok{ allbirdphen }\SpecialCharTok{\%\textgreater{}\%} \FunctionTok{filter}\NormalTok{(species }\SpecialCharTok{==} \StringTok{"bluti"}\NormalTok{)  }\CommentTok{\# first, we will select and extract cases that are of blue tits}

\NormalTok{blutiphen }\OtherTok{\textless{}{-}}\NormalTok{ blutiphen }\SpecialCharTok{\%\textgreater{}\%} \FunctionTok{select}\NormalTok{(year, site, box, fed, cs, suc)  }\CommentTok{\# we also select columns that will help us collate both databases into one by identifying individuals (i.e.: year, site and box)}

\FunctionTok{head}\NormalTok{(blutiphen)}
\end{Highlighting}
\end{Shaded}

\begin{verbatim}
##   year site box fed cs  suc
## 1 2014  FOF   1 114 11   10
## 2 2014  FOF   2 126  8    7
## 3 2014  FOF   3 111  9    6
## 4 2014  FOF   4 112 12 -999
## 5 2014  FOF   5 115  8    6
## 6 2014  FOF   6 109  7    5
\end{verbatim}

\begin{Shaded}
\begin{Highlighting}[]
\FunctionTok{nrow}\NormalTok{(allbirdphen)}
\end{Highlighting}
\end{Shaded}

\begin{verbatim}
## [1] 3594
\end{verbatim}

\begin{Shaded}
\begin{Highlighting}[]
\FunctionTok{nrow}\NormalTok{(blutiphen)  }\CommentTok{\# we end up with 2232 observations of breeding attempts}
\end{Highlighting}
\end{Shaded}

\begin{verbatim}
## [1] 2232
\end{verbatim}

\begin{Shaded}
\begin{Highlighting}[]
\CommentTok{\# Selecting variables that will be useful for the project from the adult data}

\NormalTok{adults }\OtherTok{\textless{}{-}}\NormalTok{ alladults }\SpecialCharTok{\%\textgreater{}\%} \FunctionTok{filter}\NormalTok{(season }\SpecialCharTok{!=} \StringTok{"winter"}\NormalTok{, sex }\SpecialCharTok{==} \StringTok{"F"}\NormalTok{)  }\CommentTok{\# we will remove all data coming from adults captured in the winter (as we may not have their corresponding breeding season data)}

\FunctionTok{nrow}\NormalTok{(alladults)}
\end{Highlighting}
\end{Shaded}

\begin{verbatim}
## [1] 3928
\end{verbatim}

\begin{Shaded}
\begin{Highlighting}[]
\FunctionTok{nrow}\NormalTok{(adults)  }\CommentTok{\# after this filtering, we remove 2100 observations (of male birds and of winter recordings), and we have a total of 1828 recordings of female adults ringed and identified of which we should have breeding success data in blutiphen (or, at least, of most of them)}
\end{Highlighting}
\end{Shaded}

\begin{verbatim}
## [1] 1828
\end{verbatim}

\begin{Shaded}
\begin{Highlighting}[]
\NormalTok{blutiadults }\OtherTok{\textless{}{-}}\NormalTok{ adults }\SpecialCharTok{\%\textgreater{}\%} \FunctionTok{select}\NormalTok{(ring, year, site, box, age)  }\CommentTok{\# finally, we select the columns that we will use}

\CommentTok{\# Rearranging datasets in ascending order of years, site and nest boxes}

\NormalTok{level\_order }\OtherTok{\textless{}{-}} \FunctionTok{c}\NormalTok{(}\StringTok{"EDI"}\NormalTok{, }\StringTok{"RSY"}\NormalTok{, }\StringTok{"FOF"}\NormalTok{, }\StringTok{"BAD"}\NormalTok{, }\StringTok{"LVN"}\NormalTok{, }\StringTok{"DOW"}\NormalTok{, }\StringTok{"GLF"}\NormalTok{, }\StringTok{"SER"}\NormalTok{, }\StringTok{"MCH"}\NormalTok{, }\StringTok{"PTH"}\NormalTok{, }\StringTok{"STY"}\NormalTok{, }\StringTok{"BIR"}\NormalTok{, }\StringTok{"DUN"}\NormalTok{, }\StringTok{"BLG"}\NormalTok{, }\StringTok{"PIT"}\NormalTok{, }\StringTok{"KCK"}\NormalTok{, }\StringTok{"KCZ"}\NormalTok{, }\StringTok{"BLA"}\NormalTok{, }\StringTok{"CAL"}\NormalTok{, }\StringTok{"DNM"}\NormalTok{, }\StringTok{"DNC"}\NormalTok{, }\StringTok{"DNS"}\NormalTok{, }\StringTok{"DLW"}\NormalTok{, }\StringTok{"CRU"}\NormalTok{, }\StringTok{"NEW"}\NormalTok{, }\StringTok{"HWP"}\NormalTok{, }\StringTok{"INS"}\NormalTok{, }\StringTok{"FSH"}\NormalTok{, }\StringTok{"RTH"}\NormalTok{, }\StringTok{"AVI"}\NormalTok{, }\StringTok{"AVN"}\NormalTok{, }\StringTok{"CAR"}\NormalTok{, }\StringTok{"SLS"}\NormalTok{, }\StringTok{"TOM"}\NormalTok{, }\StringTok{"DAV"}\NormalTok{, }\StringTok{"ART"}\NormalTok{, }\StringTok{"MUN"}\NormalTok{, }\StringTok{"FOU"}\NormalTok{, }\StringTok{"ALN"}\NormalTok{, }\StringTok{"DEL"}\NormalTok{, }\StringTok{"TAI"}\NormalTok{, }\StringTok{"SPD"}\NormalTok{, }\StringTok{"OSP"}\NormalTok{, }\StringTok{"DOR"}\NormalTok{)  }\CommentTok{\# here I\textquotesingle{}m creating a variable that stores sites\textquotesingle{} codes in order of increasing latitude}

\NormalTok{blutiphen }\OtherTok{\textless{}{-}}\NormalTok{ blutiphen }\SpecialCharTok{\%\textgreater{}\%} \FunctionTok{arrange}\NormalTok{(year, }\FunctionTok{factor}\NormalTok{(site, }\AttributeTok{levels =}\NormalTok{ level\_order), box) }

\NormalTok{blutiadults }\OtherTok{\textless{}{-}}\NormalTok{ blutiadults }\SpecialCharTok{\%\textgreater{}\%} \FunctionTok{arrange}\NormalTok{(year, }\FunctionTok{factor}\NormalTok{(site, }\AttributeTok{levels =}\NormalTok{ level\_order), box)  }
\end{Highlighting}
\end{Shaded}

\subsubsection{3. Creating a new database by collating blutiadults and
blutiphen using columns ``year'', ``site'' and
``box''}\label{creating-a-new-database-by-collating-blutiadults-and-blutiphen-using-columns-year-site-and-box}

\begin{Shaded}
\begin{Highlighting}[]
\CommentTok{\# First, let\textquotesingle{}s find out if there are duplicates by "year", "site" and "box" in both datasets:}
\NormalTok{blutiphen }\SpecialCharTok{\%\textgreater{}\%}
  \FunctionTok{count}\NormalTok{(year, site, box) }\SpecialCharTok{\%\textgreater{}\%}
  \FunctionTok{filter}\NormalTok{(n }\SpecialCharTok{\textgreater{}} \DecValTok{1}\NormalTok{)  }\CommentTok{\# there seem to be 5 duplicates in blutiphen}
\end{Highlighting}
\end{Shaded}

\begin{verbatim}
##   year site box n
## 1 2019  AVN   2 2
## 2 2019  EDI   1 2
## 3 2019  MUN   1 2
## 4 2019  SER   3 2
## 5 2021  DOW   4 2
\end{verbatim}

These duplicates could be second broods (within the same breeding season
and from the same female) or could be relays (if the first brood did not
success):\\
* 2019, AVN site, box 2 seems to be a relay as the first entry has no
success input\\
* 2019, EDI site, box 1 is also this case (except for the fact that in
the first entry there is a 0 instead of a NA)\\
* 2019, MUN site, box 1 also probably follows the re-lay case\\
* \ldots as also probably does 2021 site DOW, nestbox 4

\end{document}
